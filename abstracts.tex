\section*{Meeting Abstracts \& Papers}
\begin{etaremune}
\item
    \Shaoqing, SD. Zuo, \& Y. Ren. (2018).
    High-Resolution Mapping of Direct CO$_2$ Emissions and Uncertainties at the Urban Scale.
    Abstract 37 presented at the 13th International Symposium on Spatial Accuracy Assessment in Natural Resources and Environmental Sciences(Spatial Accuracy 2018), Beijing, China.
\item
    \Shaoqing\CF, XM. Zheng, SD. Zuo, \& Y. Ren. (2018).
    Improving the Prediction Accuracy of Forest Aboveground Biomass Map by Integrating Machine Learning and Spatial Statistics.
    Abstract 35 presented at the 13th International Symposium on Spatial Accuracy Assessment in Natural Resources and Environmental Sciences(Spatial Accuracy 2018), Beijing, China.
\item
    SD. Zuo, \Shaoqing, Y. Ren, \&  ZW. Yu. (2017).
    Quantifying the linear and nonlinear relations between the urban form fragmentation and the carbon emission distribution.
    Abstract GC21G-1007 presented the poster at 2017 AGU Fall Meeting, New Orleans, LA, USA.
\item
    \Shaoqing, Y. Ren, SD. Zuo, M. Dai, P. Chen, Z. Wang, LX. Xu, JW. Qi, GL. Yun. (2017).
    The environmental effect of urban form on PM2.5: a case study of Beijing-Tianjing-Hebei agglomeration.
    Presented at 9th Chinese Landscape Eoclogy workshop, Guangzhou, China.
\item
    Q. Ye, G. Zeng, \Shaoqing, FL. Wang. (2017).
     Research on the Effects of Different Policy Tools on China's Energy Conservation and Emissions Reduction Innovation Based on the Panel Data of 285 prefectural-level municipalities.
     Presented at 2017 annual conference of the Chinese geographical society, economic geography and Specialized Committee.Sichuan, China.
\item
    SD. Zuo, \Shaoqing, Y. Ren, \&  ZW. Yu. (2017).
    Quantifying the linear and nonlinear relations between the urban form fragmentation and the carbon emission distribution.
    Presented the poster at 9th Lecture of Modern Ecology, Shanghai, China.
\item
    \Shaoqing, HX. Jiang, JJ. Li, X. Su, J. Wu. (2016).
    The Criminal Geographical Analysis about Walking Environment of Urban.
    Presented at The Workshop of 12th Spatial Behavior and Planning \& Spatial-Temporal Behavior and Social Planning Research, Beijing, China.
\item
    \Shaoqing, HX. Jiang, JJ. Li, QW. Xu. (2016).
    Transportation Planning of Smart City on the Basis of Data Augmentation Design Under the perspective of Humanism: A Case Study of Fuzhou’s Cangshan District.
    Abstract 1813-2948-1 presented at The 10th annual conference of International Association of China Planning, Beijing, China.
\item
    \Shaoqing, HX. Jiang, JJ. Li. (2016).
    The Simulation Analysis of Waterlogging and the Sponge City Planning Control of Central Urabn Area in Fuzhou City.
    Abstract 391 presented the poster at 2016 International Low Impact Development Conference, Beijing, China.
\end{etaremune}
