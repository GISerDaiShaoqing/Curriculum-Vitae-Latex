% AGU style: https://publications.agu.org/agu-grammar-and-style-guide/
\newcommand{\Revision}{\textit{under revision}}
\newcommand{\Review}{\textit{under review}}
\newcommand{\Submitted}{\textit{submitted}}
\newcommand{\CS}{*} % corresponding author
\newcommand{\CF}{\textsuperscript{\#}} % co-first author

\section{\texorpdfstring{\faBook\ Publications}{Publications}}
\CS corresponding author, \CF co-first author.

\subsection*{\texorpdfstring{\faBook\ Peer-Reviewed Journal Articles}{Peer-Reviewed Journal Articles}}
%\subsection*{Peer-Reviewed Journal Articles}
%\begin{enumerate}
\begin{etaremune}
\item
    GL. Yun, YR. He, YT. Jiang, PF. Dou, \Shaoqing. (2019).
    PM2.5 spatiotemporal evolution and drivers in the Yangtze River Delta between 2005 and 2015.
    \textit{Atmosphere}. \textit{10}(2), 55-69
    \DOI{10.3390/atmos10020055}. [SCI, IF = 1.704]
\item
    SD. Zuo, \Shaoqing, XD. Song, CD. Xu, YL. Liao, WY. Chang, Q. Chen, YY. Li, JF. Tang, W. Man, Y. Ren. (2018).
    Determining the Mechanisms that Influence the Surface Temperature of Urban Forest Canopies by Combining Remote Sensing Methods, Ground Observations, and Spatial Statistical Models. 
    \textit{Remote Sensing}. \textit{10}(11), 1814-1832
    \DOI{10.3390/rs10111814}. [SCI, IF = 3.406]
\item
    SD. Zuo, \Shaoqing, YY. Li, JF. Tang, Y. Ren. (2018).
	Analysis of Heavy Metal Sources in the Soil of Riverbanks Across an Urbanization Gradient.
    \textit{International Journal of Environmental Research and Public Health}. \textit{15}(10), 2175-2198
    \DOI{10.3390/ijerph15102175}. [SCI, IF = 2.145]
\item
    \Shaoqing, HX. Jiang, JJ. Li, X. Su, J. Wu, Y. Ren. (2018).
	Influence of walking environment on robbery, snatch and theft crime in urban area, H city, China.
    \textit{Scientia Geographica Sinica}.  \textit{38}(8), 1235-1244
    \DOI{10.13249/j.cnki.sgs.2018.08.005}. [CSCD, in Chinese]
\item
    GL. Yun, SD. Zuo, \Shaoqing, XD. Song, CD. Xu, YL. Liao, PQ. Zhao, WY. Chang, Q. Chen, YY. Li, JF. Tang, W. Man, Y. Ren. (2018).
	Individual and Interactive Influences of Anthropogenic and Ecological Factors on Forest PM2.5 Concentrations at an Urban Scale.
    \textit{Remote Sensing}. \textit{10}(4), 521-538
    \DOI{10.3390/rs10040521}. [SCI, IF = 3.406]
\item
    Q. Ye, G. Zeng, \Shaoqing, FL. Wang. (2018).
	Research on the effects of different policy tools on China’s emissions reduction innovation.
    \textit{China Population, Resources and Environment}. \textit{210}(02), 115-122.
    \DOI{10.12062/cpre.20170915}. [CSCD, in Chinese]
\item
    SD. Zuo, \Shaoqing, Y. Ren, ZW. Yu. (2017).
    Quantifying the linear and nonlinear relations between the urban form fragmentation and the carbon emission distribution.
    \textit{American Geophysical Union, Fall Meeting 2017, abstract GC21G-1007}.
\item
    Q. Ye, \Shaoqing, G. Zeng. (2017).
	Research on the effects of command-and-control and market-oriented policy tools on China’s energy conservation and emissions reduction innovation.
    \textit{Chinese Journal of Population Resources and Environment}. \textit{16}(1), 1-11.
    \DOI{10.1080/10042857.2017.1418273}. [ESCI]
\item
    ML. Li, \Shaoqing, JY. Wang, ZJ. Shen. (2016).
	The Analysis of Urban Spatial Development Pattern in Beijing Based on the Big Data of Government.
    \textit{Geomatics World}. \textit{23}(3), 20-26.
    \DOI{10.3969/j.issn.1672-1586.2016.03.004}. [in Chinese]
\end{etaremune}
%\end{enumerate}

%\subsection*{Book Chapter}
\subsection*{\texorpdfstring{\faBook\ Book Chapter}{Book Chapter}}
%\begin{enumerate}
\begin{etaremune}
\item
    GL. Yun, SD. Zuo, \Shaoqing, XD. Song, CD. Xu, YL. Liao, PQ. Zhao, WY. Chang, Q. Chen, YY. Li, JF. Tang, W. Man, Y. Ren. (2019).
	Individual and Interactive Influences of Anthropogenic and Ecological Factors on Forest PM2.5 Concentrations at an Urban Scale.
    \textit{Advances in Quantitative Remote Sensing in China–In Memory of Prof. Xiaowen Li}.316-332.
\item
    \Shaoqing, JJ. Li, SD. Zuo, Y. Ren, HX. Jiang. (2017).
	Landscape-Scale Simulation Analysis of Waterlogging and Sponge City Planning for a Central Urban Area in Fuzhou City, China.
    \textit{International Low Impact Development Conference China 2016 : LID Applications in Sponge City Projects}. 251-260.
    \DOI{10.1061/9780784481042.028}. [EI]
\end{etaremune}
%\end{enumerate}

%\subsection*{Papers submitted/under revision}
\subsection*{\texorpdfstring{\faBook\ Papers submitted/under revision}{Papers submitted/under revision}}
%\begin{enumerate}
\begin{etaremune}
\item
    \Shaoqing, SD. Zuo, \& Y. Ren.
    High-resolution mapping of direct CO$_2$ emissions and uncertainties at the urban scale. 
    (\Submitted).
\item
    \Shaoqing\CF, XM. Zheng, SD. Zuo, \& Y. Ren.
    Improving the Prediction Accuracy of Forest Aboveground Biomass Map by Integrating Machine Learning and Spatial Statistics. 
    (\Submitted).
\item
    SD. Zuo, \Shaoqing, Y. Ren, R. Zhou, \&  ZW. Yu.
    More fragmentized urban form more CO$_2$ emissions? An comprehensive relationship from the combination analysis across different scales 
    (\Revision).
\item
    Q. Yang, TY. Huang, SG. Wang, JS. Li, \Shaoqing, S. Wright, YX. Wang, \& HW. Peng.
    A GIS-based high spatial resolution assessment of large-scale PV power generation potential and associated CO$_2$ emission reduction in China.
    (\Revision).
\end{etaremune}
%\end{enumerate}
\subsection*{\texorpdfstring{\faBook\ Patent}{Patent}}
\begin{etaremune}
\item
     Q. Chen, Y. Ren, XM. Zheng, SD. Zuo. \Shaoqing.
     A mixed-effect model for estimation of large area subtropical forest biomass.
     (\textit{under apply}).
\end{etaremune}