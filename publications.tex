% AGU style: https://publications.agu.org/agu-grammar-and-style-guide/
\newcommand{\Revision}{\textit{under revision}}
\newcommand{\Review}{\textit{under review}}
\newcommand{\Submitted}{\textit{submitted}}
\newcommand{\CS}{*} % corresponding author
\newcommand{\CF}{\textsuperscript{\#}} % co-first author

\section{\texorpdfstring{\faBook\ Publications}{Publications}}
\CS corresponding author, \CF co-first author.

\subsection*{\texorpdfstring{\faBook\ Peer-Reviewed Journal Articles}{Peer-Reviewed Journal Articles}}
%\subsection*{Peer-Reviewed Journal Articles}
\begin{enumerate}
%\begin{etaremune}
\item
    SD. Zuo, LP. Yang, PF. Dou, H. Chak Ho, \Shaoqing, WJ. Ma, Y. Ren, CR. Huang. (2020).
    The direct and interactive impacts of hydrological factors on bacillary dysentery across different geographical regions in central China.
    \textit{Science of the Total Environment}. \textit{764}, 144609.
    \DOI{10.1016/j.scitotenv.2020.144609}. [SCI, IF = 6.551].
\item
    \Shaoqing\CF, Y. Ren, SD. Zuo, CY. Lai, JJ. Li, SY. Xie, BC. Chen. (2020).
    Investigating the uncertainties propagation analysis of CO$_2$ emissions gridded maps at the urban scales: a case study of Jinjiang city, China. 
    \textit{Remote Sensing}. \textit{12}(23), 3932.
    \DOI{10.3390/rs12233932}. [SCI, IF = 4.509]
\item
    \Shaoqing, JJ. Li, WE. Yang, FY. Chen, HX. Jiang. (2020).
    The Evaluation of Health Effect of Short-term Exposure to PM2.5 during Spring Festival: A Case Study of 25 Cities in the Yangtze River Delta.
    \textit{Shanghai Urban Planning Review}. \textit{05}, 22-29.
    \DOI{10.11982/j.supr.20200504}. [in Chinese] 
\item
    P. Jia, \Shaoqing,K. E. Rohli, R.V. Rohli, YN. Ma, C. Yu, XF. Pan, WQ. Zhou. (2020).
    Natural environment and childhood obesity: A systematic review.
    \textit{Obesity Reviews}. 1-9.
    \DOI{10.1111/obr.13097}. [SCI, IF = 7.31]
\item
    XF. Pan, H. Li, LX. Zhao, XR. Yang, JQ. Su, \Shaoqing, J. Ning, CX. Li, GJ. Cai, GF. Zhu. (2020).
    Response of syntrophic bacterial and methanogenic archaeal communities in paddy soil to soil type and phenological period of rice growth.
    \textit{Journal of Cleaner Production}. \textit{278}, 123418.
    \DOI{10.1016/j.jclepro.2020.123418}. [SCI, IF = 7.246]
\item
    P. Jia, XX. Cao, HX. Yang, \Shaoqing, P. He, GL. Huang, T. Wu, YG. Wang. (2020).
    Green space access in the neighbourhood and childhood obesity.
    \textit{Obesity Reviews}. 1-12.
    \DOI{10.1111/obr.13100}. [SCI, IF = 7.31]
\item
    MY. Luo, HQ. Li, XF. Pan, T. Fei, \Shaoqing, Q. Ge, YX. Zou, H. Vos, JY. Luo, P. Jia. (2020).
    Neighbourhood speed limit and childhood obesity.
    \textit{Obesity Reviews}. 1-11.
    \DOI{10.1111/obr.13052}. [SCI, IF = 7.31]
\item
    SY. Xie, GW. Yu, CX. Li, J. Li, G. Wang, \Shaoqing, Y. Wang. (2020).
    Treatment of high-ash industrial sludge for producing improved char with low heavy metal toxicity.
    \textit{Journal of Analytical and Applied Pyrolysis}. \textit{150}, 104866.
    \DOI{10.1016/j.jaap.2020.104866}. [SCI, IF = 3.905]
\item
    XF. Pan, L. Zhao, JY. Luo, YH. Li, L.Zhang, T. Wu, M. Smith, \Shaoqing, P. Jia. (2020).
    Access to bike lanes and childhood obesity: A systematic review and meta‐analysis.
    \textit{Obesity Reviews}. 1-11.
    \DOI{10.1111/obr.13042}. [SCI, IF = 7.31]
\item
    \Shaoqing, SD. Zuo, Y. Ren. (2020).
    A spatial database of CO$_2$ emissions, urban form fragmentation and city-scale effect related impact factors for the low carbon urban system in Jinjiang city, China.
    \textit{Data in Brief}. \textit{29}, 105274.
    \DOI{10.1016/j.dib.2020.105274}. [ESCI, CiteScore = 1.5]
\item
    SD. Zuo, \Shaoqing, Y. Ren. (2019).
    More fragmentized urban form more CO$_2$ emissions? An comprehensive relationship from the combination analysis across different scales.
    \textit{Journal of Cleaner Production}. \textit{244}, 118659.
    \DOI{10.1016/j.jclepro.2019.118659}. [SCI, IF = 7.246]
\item
    PF. Dou, SD. Zuo, Y. Ren, \Shaoqing, GL. Yun. (2019).
    The impacts of climate and land use/land cover changes on water yield service in Ningbo region.
    \textit{Acta Scientiae Circumstantiae}. \textit{39}(7), 2398-2409.
    \DOI{10.13671/j.hjkxxb.2019.0122}. [CSCD, in Chinese] 
\item
    Q. Yang, TY. Huang, SG. Wang, JS. Li, \Shaoqing, S. Wright, YX. Wang, \& HW. Peng.(2019).
    A GIS-based high spatial resolution assessment of large-scale PV power generation potential and associated CO$_2$ emission reduction in China.
    \textit{Applied Energy}. \textit{247}, 254-269.
    \DOI{10.1016/j.apenergy.2019.04.005}. [SCI, IF = 8.848]
\item
    GL. Yun, YR. He, YT. Jiang, PF. Dou, \Shaoqing. (2019).
    PM2.5 spatiotemporal evolution and drivers in the Yangtze River Delta between 2005 and 2015.
    \textit{Atmosphere}. \textit{10}(2), 55.
    \DOI{10.3390/atmos10020055}. [SCI, IF = 2.397]
\item
    SD. Zuo, \Shaoqing, XD. Song, CD. Xu, YL. Liao, WY. Chang, Q. Chen, YY. Li, JF. Tang, W. Man, Y. Ren. (2018).
    Determining the Mechanisms that Influence the Surface Temperature of Urban Forest Canopies by Combining Remote Sensing Methods, Ground Observations, and Spatial Statistical Models. 
    \textit{Remote Sensing}. \textit{10}(11), 1814.
    \DOI{10.3390/rs10111814}. [SCI, IF = 4.509]
\item
    SD. Zuo, \Shaoqing, YY. Li, JF. Tang, Y. Ren. (2018).
	Analysis of Heavy Metal Sources in the Soil of Riverbanks Across an Urbanization Gradient.
    \textit{International Journal of Environmental Research and Public Health}. \textit{15}(10), 2175.
    \DOI{10.3390/ijerph15102175}. [SCI, IF = 2.849]
\item
    \Shaoqing, HX. Jiang, JJ. Li, X. Su, J. Wu, Y. Ren. (2018).
	Influence of walking environment on robbery, snatch and theft crime in urban area, H city, China.
    \textit{Scientia Geographica Sinica}.  \textit{38}(8), 1235-1244.
    \DOI{10.13249/j.cnki.sgs.2018.08.005}. [CSCD, in Chinese]
\item
    GL. Yun, SD. Zuo, \Shaoqing, XD. Song, CD. Xu, YL. Liao, PQ. Zhao, WY. Chang, Q. Chen, YY. Li, JF. Tang, W. Man, Y. Ren. (2018).
	Individual and Interactive Influences of Anthropogenic and Ecological Factors on Forest PM2.5 Concentrations at an Urban Scale.
    \textit{Remote Sensing}. \textit{10}(4), 521.
    \DOI{10.3390/rs10040521}. [SCI, IF = 4.509]
\item
    Q. Ye, G. Zeng, \Shaoqing, FL. Wang. (2018).
	Research on the effects of different policy tools on China’s emissions reduction innovation.
    \textit{China Population, Resources and Environment}. \textit{28}(02), 115-122.
    \DOI{10.12062/cpre.20170915}. [CSCD, in Chinese]
\item
    Q. Ye, \Shaoqing, G. Zeng. (2017).
	Research on the effects of command-and-control and market-oriented policy tools on China’s energy conservation and emissions reduction innovation.
    \textit{Chinese Journal of Population Resources and Environment}. \textit{16}(1), 1-11.
    \DOI{10.1080/10042857.2017.1418273}. [ESCI, CiteScore = 1.9]
\item
    MY. Li, \Shaoqing, JY. Wang, ZJ. Shen. (2016).
	The Analysis of Urban Spatial Development Pattern in Beijing Based on the Big Data of Government.
    \textit{Geomatics World}. \textit{23}(3), 20-26.
    \DOI{10.3969/j.issn.1672-1586.2016.03.004}. [in Chinese]
%\end{etaremune}
\end{enumerate}

%\subsection*{Book Chapter}
\subsection*{\texorpdfstring{\faBook\ Book Chapter}{Book Chapter}}
\begin{enumerate}
%\begin{etaremune}
\item
    GL. Yun, SD. Zuo, \Shaoqing, XD. Song, CD. Xu, YL. Liao, PQ. Zhao, WY. Chang, Q. Chen, YY. Li, JF. Tang, W. Man, Y. Ren. (2019).
	Individual and Interactive Influences of Anthropogenic and Ecological Factors on Forest PM2.5 Concentrations at an Urban Scale.
    \textit{Advances in Quantitative Remote Sensing in China–In Memory of Prof. Xiaowen Li}.316-332.
\item
    \Shaoqing, JJ. Li, SD. Zuo, Y. Ren, HX. Jiang. (2017).
	Landscape-Scale Simulation Analysis of Waterlogging and Sponge City Planning for a Central Urban Area in Fuzhou City, China.
    \textit{International Low Impact Development Conference China 2016 : LID Applications in Sponge City Projects}. 251-260.
    \DOI{10.1061/9780784481042.028}. [EI]
%\end{etaremune}
\end{enumerate}

\subsection*{\texorpdfstring{\faBook\ Patent}{Patent}}
%\begin{etaremune}
\begin{enumerate}
\item
     Y. Ren, XM. Zheng, \Shaoqing, Q.Chen, SD. Zuo.
     Improving the accuracy of forest biomass estimate using source analysis and machine learning.
     (\Review).
\item
     Y. Ren, \Shaoqing, SD. Zuo, XM. Zheng.
     A forest inventory biomass estimation model by multi-sources data fusion.
     (\Review).
\item
     Q. Chen, Y. Ren, XM. Zheng, SD. Zuo, \Shaoqing.
     A mixed-effect model for estimation of large area subtropical forest biomass.
     (\Review).
%\end{etaremune}
\end{enumerate}

\subsection*{\texorpdfstring{\faBook\ Software Copyright}{Software Copyright}}
\begin{enumerate}
\item
    A water saftety monitation system based on environmental IOT data and InVEST model. V1.0.
    \textit{RN: 2019SR0517519}. (2019).
\item
    A water conservation service monitation system based on environmental IOT data. V1.0.
    \textit{RN: 2018SR745897}. (2018).
\end{enumerate}

\subsection*{\texorpdfstring{\faBook\ Conference Paper \& Abstract}{Conference Paper \& Abstract}}
\begin{enumerate}
\item
    \Shaoqing, JJ. Li, WE. Yang, FY. Chen, HX. Jiang. (2020).
    The Evaluation of Health Effect of Short-term Exposure to PM2.5 during Spring Festival: A Case Study of 25 Cities in the Yangtze River Delta.
    \textit{The Workshop of 16th Spatial Behavior and Planning \& Digitalization and Fine-Grained Governance, abstract}
\item
    JJ. Li, YP. Liu, \Shaoqing, KY. Xiang, HX. Jiang, WQ. Chen. (2019).
    The night light uncovers city’s weight—A case study on estimating construction materials in Fuzhou, China.
    \textit{10th International Conference on Industrial Ecology, abstract 320}
\item
    \Shaoqing. (2018).
    The Scale Effect, Zoning Effect of Geographical Features, the Modified Areal Unit Problem and the Chanlleges of Spatial Stastistics.
    \textit{11th Chinese R Language Conference, abstract}
\item
    \Shaoqing, SD. Zuo, Y. Ren. (2018).
    High-resolution mapping of direct CO$_2$ emissions and uncertainties at the urban scale.
    \textit{Proceedings of Spatial Accuracy 2018}. 88-90. [EI]
\item
    \Shaoqing\CF, XM. Zheng, SD. Zuo,Y. Ren. (2018).
    Improving the prediction accuracy of forest aboveground biomass benchmark map by integrating machine learning and spatial statistics.
    \textit{Proceedings of Spatial Accuracy 2018}. 80-83. [EI]
\item
    SD. Zuo, \Shaoqing, Y. Ren, ZW. Yu. (2017).
    Quantifying the linear and nonlinear relations between the urban form fragmentation and the carbon emission distribution.
    \textit{American Geophysical Union, Fall Meeting 2017, abstract GC21G-1007}.
\item
    \Shaoqing, Y.Ren, SD. Zuo, M. Dai, P. Chen, Z. Wang, LX. Xu, JW. Qi, GL. Yun. (2017).
    The Environmental Effect of Urban Form on PM2.5: A Case Study of Beijing-Tianjin-Hebei Urban Agglomerations.
    \textit{9th Chinese Landscape Eoclogy workshop, abstract}.
\item
    Q. Ye, G. Zeng, \Shaoqing, FL. Wang. (2017).
    Research on the Effects of Different Policy Tools on China’s Energy Conservation and Emissions Reduction Innovation Based on the Panel Data of 285 prefectural-level municipalities.
    \textit{2017 annual conference of the Chinese geographical society, economic geography and Specialized Committee Abstracts Proceedings, abstract}.
\item
    \Shaoqing, Y.Ren, SD. Zuo. (2017).
    The relationship between the fragmented urban form landscape and the carbon emission distribution at different resolutions.
    \textit{9th Lecture of Modern Ecology, abstract}.
\item
    \Shaoqing, HX. Jiang, JJ. Li, X. Su, J. Wu, K. Chen. (2016).
    The environmental crime analysis based on WalkScore and CGT model, a case study of H city.
    \textit{The Workshop of 12th Spatial Behavior and Planning \& Spatial-Temporal Behavior and Social Planning Researchg, abstract}.
\item
    \Shaoqing, HX. Jiang, JJ. Li, QW. Xu. (2016).
    Transportation Planning of Smart City on the Basis of Data Augmentation Design Under the Perspective of Humanism: A Case Study of Fuzhou's Cangshan District.
    \textit{10th annual conference of International Association of China Planning, abstract 1813}.
\item
   \Shaoqing, HX. Jiang, JJ. Li. (2016).
    The Simulation Analysis of Waterlogging and the Sponge City Planning Control of Central Urabn Area in Fuzhou City.
    \textit{2016 International Low Impact Development Conference, paper 391}.
\end{enumerate}

\subsection*{\texorpdfstring{\faBook\ Open Dataset \& Code}{Open Dataset \& Code}}
\begin{enumerate}
\item
   \Shaoqing. (2020).
    GISerDaiShaoqing/3dmodelFJNU 1.0 (Version 1.0).
    \textit{[Data set]}. Zenodo. 
    \DOI{10.5281/zenodo.4277977}.
\item
   \Shaoqing, SD. Zuo, Y.Ren. (2020).
    GISerDaiShaoqing/Urban-Carbon-Dioxide-sources-gridded-maps-and-its-detemination-in-Jinjiang-city 0.6 (Version 0.6) .
    \textit{[Data set]}. Zenodo. 
    \DOI{10.5281/zenodo.3566072}.
\end{enumerate}

%\subsection*{Papers submitted/under revision}
\subsection*{\texorpdfstring{\faBook\ Papers submitted/under revision}{Papers submitted/under revision}}
\begin{enumerate}
%\begin{etaremune}
\item
    \Shaoqing\CF, XM. Zheng, L. Gao, CD. Xu, SD. Zuo, Q. Chen, XH. Wei, Y. Ren.
    Improving plot-level model of forest aboveground biomass: A combined approach using machine learning with spatial statistics.
    \textit{Journal of Cleaner Production}. 
    (\Review).
\item
    XM. Zheng, \Shaoqing, Q. Chen, K. Calders, CD. Xu, XH. Wei, MZ. Zhuang, Y. Ren.
    Integrating Spatial Statistics and Machine Learning to Analyze Error Sources in Estimating Forest Aboveground Biomass from Airborne LiDAR Data.
    \textit{Agricultural and Forest Meteorology}. 
    (\Revision).
\item
    LL. Song, \Shaoqing, Z. Cao, YP. Liu, WQ. Chen.
    Spatiotemporal dynamics of steel resources accumulated above ground in mainland China: 1995-2030.
    \textit{Journal of Cleaner Production}. 
    (\Review).
\item
    SJ. Yang, \Shaoqing, YL Huang, P. Jia.
    Commentary: Substantial undocumented infection facilitates the rapid dissemination of novel coronavirus (SARS-CoV2).
    \textit{Frontiers in Public Health}. 
    (\Review).
\item
    PF. Dou, SD.Zuo, Y. Ren, M.J. Rodriguez, \Shaoqing.
    Refined water security assessment for sustainable water management: A case study of 15 key cities in the Yangtze River Delta, China.
    \textit{Journal of Hydrology}.
    (\Revision).
%\end{etaremune}
\end{enumerate}