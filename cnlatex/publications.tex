% AGU style: https://publications.agu.org/agu-grammar-and-style-guide/
\newcommand{\Revision}{\songtix{返修}}
\newcommand{\Review}{\songtix{审稿}}
\newcommand{\Submitted}{\songtix{投稿}}
\newcommand{\CS}{*} % corresponding author
\newcommand{\CF}{\textsuperscript{\#}} % co-first author


\section{\texorpdfstring{\faBook\ 出版物}{出版物}}
\CS 通讯作者, \CF 共同第一作者.

\subsection*{\texorpdfstring{\faBook\ 期刊论文}{期刊论文}}
%\subsection*{Peer-Reviewed Journal Articles}
\begin{enumerate}
%\begin{etaremune}
\item
    \Shaoqing, SD. Zuo, Y. Ren. (2020).
    A spatial database of CO$_2$ emissions, urban form fragmentation and city-scale effect related impact factors for the low carbon urban system in Jinjiang city, China.
    \textit{Data in Brief}. \textit{Accepted}.
    \DOI{10.1016/j.dib.2020.105274}. [CiteScore = 0.93]
\item
    SD. Zuo, \Shaoqing, Y. Ren. (2020).
    More fragmentized urban form more CO$_2$ emissions? An comprehensive relationship from the combination analysis across different scales.
    \textit{Journal of Cleaner Production}. \textit{244}.
    \DOI{10.1016/j.jclepro.2019.118659}. [SCI, IF = 6.395]
\item
    窦攀烽, 左舒翟, 任引, {\heiti{戴劭勍}}, 云国梁. (2019).
    气候和土地利用/覆被变化对宁波地区生态系统产水服务的影响.
    {\songtix{环境科学学报}}. \textit{39}(7), 2398-2409.
    \DOI{10.13671/j.hjkxxb.2019.0122}. [CSCD, 复合影响因子=2.703] 
\item
    Q. Yang, TY. Huang, SG. Wang, JS. Li, \Shaoqing, S. Wright, YX. Wang, \& HW. Peng.(2019).
    A GIS-based high spatial resolution assessment of large-scale PV power generation potential and associated CO$_2$ emission reduction in China.
    \textit{Applied Energy}. \textit{247}, 254-269.
    \DOI{10.1016/j.apenergy.2019.04.005}. [SCI, IF = 8.426]
\item
    GL. Yun, YR. He, YT. Jiang, PF. Dou, \Shaoqing. (2019).
    PM2.5 spatiotemporal evolution and drivers in the Yangtze River Delta between 2005 and 2015.
    \textit{Atmosphere}. \textit{10}(2), 55.
    \DOI{10.3390/atmos10020055}. [SCI, IF = 2.046]
\item
    SD. Zuo, \Shaoqing, XD. Song, CD. Xu, YL. Liao, WY. Chang, Q. Chen, YY. Li, JF. Tang, W. Man, Y. Ren. (2018).
    Determining the Mechanisms that Influence the Surface Temperature of Urban Forest Canopies by Combining Remote Sensing Methods, Ground Observations, and Spatial Statistical Models. 
    \textit{Remote Sensing}. \textit{10}(11), 1814.
    \DOI{10.3390/rs10111814}. [SCI, IF = 4.118]
\item
    SD. Zuo, \Shaoqing, YY. Li, JF. Tang, Y. Ren. (2018).
	Analysis of Heavy Metal Sources in the Soil of Riverbanks Across an Urbanization Gradient.
    \textit{International Journal of Environmental Research and Public Health}. \textit{15}(10), 2175.
    \DOI{10.3390/ijerph15102175}. [SCI, IF = 2.468]
\item
    {\heiti{戴劭勍}}, 江辉仙, 李佳佳, 苏娴, 吴娟, 任引. (2018).
	H市城区步行环境对两抢一盗警情的影响.
    {\songtix{地理科学}}.  \textit{38}(8), 1235-1244.
    \DOI{10.13249/j.cnki.sgs.2018.08.005}. [CSCD, 复合影响因子=4.803]
\item
    GL. Yun, SD. Zuo, \Shaoqing, XD. Song, CD. Xu, YL. Liao, PQ. Zhao, WY. Chang, Q. Chen, YY. Li, JF. Tang, W. Man, Y. Ren. (2018).
	Individual and Interactive Influences of Anthropogenic and Ecological Factors on Forest PM2.5 Concentrations at an Urban Scale.
    \textit{Remote Sensing}. \textit{10}(4), 521.
    \DOI{10.3390/rs10040521}. [SCI, IF = 4.118]
\item
    叶琴, 曾刚, {\heiti{戴劭勍}}, 王丰龙. (2018).
	不同环境规制工具对中国节能减排技术创新的影响——基于285个地级市面板数据.
    {\songtix{中国人口·资源与环境}}. \textit{28}(02), 115-122.
    \DOI{10.12062/cpre.20170915}. [CSCD, 复合影响因子=5.428]
\item
    Q. Ye, \Shaoqing, G. Zeng. (2017).
	Research on the effects of command-and-control and market-oriented policy tools on China’s energy conservation and emissions reduction innovation.
    \textit{Chinese Journal of Population Resources and Environment}. \textit{16}(1), 1-11.
    \DOI{10.1080/10042857.2017.1418273}. [ESCI]
\item
   李苗裔, {\heiti{戴劭勍}}, 王静远, 沈振江. (2016).
	基于政府大数据的北京城市空间发展模式分析.
    {\songtix{地理信息世界}}. \textit{23}(3), 20-26.
    \DOI{10.3969/j.issn.1672-1586.2016.03.004}. [JST, 复合影响因子=1.026]
%\end{etaremune}
\end{enumerate}

%\subsection*{Book Chapter}
\subsection*{\texorpdfstring{\faBook\ 著作章节}{著作章节}}
\begin{enumerate}
%\begin{etaremune}
\item
    GL. Yun, SD. Zuo, \Shaoqing, XD. Song, CD. Xu, YL. Liao, PQ. Zhao, WY. Chang, Q. Chen, YY. Li, JF. Tang, W. Man, Y. Ren. (2019).
	Individual and Interactive Influences of Anthropogenic and Ecological Factors on Forest PM2.5 Concentrations at an Urban Scale.
    \textit{Advances in Quantitative Remote Sensing in China–In Memory of Prof. Xiaowen Li}.316-332.
\item
    \Shaoqing, JJ. Li, SD. Zuo, Y. Ren, HX. Jiang. (2017).
	Landscape-Scale Simulation Analysis of Waterlogging and Sponge City Planning for a Central Urban Area in Fuzhou City, China.
    \textit{International Low Impact Development Conference China 2016 : LID Applications in Sponge City Projects}. 251-260.
    \DOI{10.1061/9780784481042.028}. [EI]
%\end{etaremune}
\end{enumerate}

%\subsection*{Papers submitted/under revision}
\subsection*{\texorpdfstring{\faBook\ 投稿/返修论文}{投稿/返修论文}}
\begin{enumerate}
%\begin{etaremune}
\item
    \Shaoqing, SD. Zuo, Y. Ren.
    A framework for uncertainty propagation in gridded maps of CO$_2$ emissions: a case study of Jinjiang city, China.  
   \textit{Scientific Reports}.
    (\Review).
\item
    \Shaoqing\CF, XM. Zheng, L. Gao, SD. Zuo, Q. Chen, XH Wei, Y. Ren.
    Improving prediction of forest aboveground biomass maps: A combined approach of machine learning with spatial statistical model.
    \textit{Biogeosciences}. 
    (\Review).
\item
    PF. Dou, SD.Zuo, Y. Ren, M.J. Rodriguez, \Shaoqing.
    Refined water security assessment for sustainable water management: A case study of 15 key cities in the Yangtze River Delta, China.
    \textit{Journal of Hydrology}.
    (\Review).
\item
    XF. Pan, H. Li, XR. Yang, JQ. Su, \Shaoqing, CX. Li, GJ. Cai, GF. Zhu.
    Spatio-temporal distribution of syntrophs in paddy soils from Southern China.
    \textit{Geoderma}.
    (\Review).
\item
    SY. Xie, CX. Li, J. Li, G. Wang, \Shaoqing, Y. Wang, GW. Yu.
    Treatment of industrial sludge via co-pyrolysis with rice straw for biochar improvement and heavy metals immobilization.
    \textit{Journal of Cleaner Production}. 
    (\Review).
%\end{etaremune}
\end{enumerate}
\subsection*{\texorpdfstring{\faBook\ 专利}{专利}}
%\begin{etaremune}
\begin{enumerate}
\item
     任引, {\heiti{戴劭勍}}, 左舒翟, 郑小曼.
     一种多源数据融合的森林资源清查生物量估算模型.
     ({\songtix{实审}}).
\item
     陈奇, 任引, 郑小曼, 左舒翟, {\heiti{戴劭勍}}.
     一种预测大面积亚热带森林生物量的混合效应模型.
     ({\songtix{实审}}).
%\end{etaremune}
\end{enumerate}

\subsection*{\texorpdfstring{\faBook\ 软件著作权}{软件著作权}}
\begin{enumerate}
\item
   基于环境物联网与 InVEST 的水安全监测系统. V1.0.
    \textit{登记号: 2019SR0517519}. (2019).
\item
    基于环境物联网数据的水源涵养监测系统. V1.0.
    \textit{登记号: 2018SR745897}. (2018).
\end{enumerate}

\subsection*{\texorpdfstring{\faBook\ 会议论文 \& 摘要}{会议论文 \& 摘要}}
\begin{enumerate}
\item
    JJ. Li, YP. Liu, \Shaoqing, KY. Xiang, HX. Jiang, WQ. Chen. (2019).
    The night light uncovers city’s weight—A case study on estimating construction materials in Fuzhou, China.
    \textit{10th International Conference on Industrial Ecology, abstract 320}
\item
    {\heiti{戴劭勍}}. (2018).
    地理要素的尺度,分区效应,可变面积单元问题与空间统计的挑战.
    \textit{第十一届中国R语言会议, abstract}
\item
    \Shaoqing, SD. Zuo, Y. Ren. (2018).
    High-resolution mapping of direct CO$_2$ emissions and uncertainties at the urban scale.
    \textit{Proceedings of Spatial Accuracy 2018}. 88-90. [EI]
\item
    \Shaoqing\CF, XM. Zheng, SD. Zuo,Y. Ren. (2018).
    Improving the prediction accuracy of forest aboveground biomass benchmark map by integrating machine learning and spatial statistics.
    \textit{Proceedings of Spatial Accuracy 2018}. 80-83. [EI]
\item
    SD. Zuo, \Shaoqing, Y. Ren, ZW. Yu. (2017).
    Quantifying the linear and nonlinear relations between the urban form fragmentation and the carbon emission distribution.
    \textit{American Geophysical Union, Fall Meeting 2017, abstract GC21G-1007}.
\item
    {\heiti{戴劭勍}}, 任引, 左舒翟, 代敏, 陈盼, 王朝, 徐凌星, 祁建伟, 云国梁. (2017).
    城市形态的PM2.5环境效应——以京津冀城市群为例.
    \textit{第九届中国景观生态学学术研讨会, abstract}.
\item
    叶琴, 曾刚, {\heiti{戴劭勍}}, 王丰龙. (2017).
    不同环境规制工具对节能减排技术创新的影响——基于中国285个地级市的面板数据.
    \textit{2017年中国地理学会经济地理专业委员会学术年会论文摘要集, abstract}.
\item
    {\heiti{戴劭勍}}, 任引, 左舒翟. (2017).
    不同分辨率的城市碳排放空间分布与城市空间形态景观破碎化之间的关系.
    \textit{第九届现代生态学讲座, abstract}.
\item
    {\heiti{戴劭勍}}, 江辉仙, 李佳佳, 苏娴, 吴娟, 陈焜. (2016).
    基于WalkScore与CGT模型的环境犯罪学分析:以H市为例.
    \textit{时空间行为与城市社会规划研究暨第十二次空间行为与规划研讨会, abstract}.
\item
    \Shaoqing, HX. Jiang, JJ. Li, QW. Xu. (2016).
    Transportation Planning of Smart City on the Basis of Data Augmentation Design Under the Perspective of Humanism: A Case Study of Fuzhou's Cangshan District.
    \textit{10th annual conference of International Association of China Planning, abstract 1813}.
\item
   \Shaoqing, HX. Jiang, JJ. Li. (2016).
    The Simulation Analysis of Waterlogging and the Sponge City Planning Control of Central Urabn Area in Fuzhou City.
    \textit{2016 International Low Impact Development Conference, paper 391}.
\end{enumerate}

\subsection*{\texorpdfstring{\faBook\ 开放数据 \& 代码}{开放数据 \& 代码}}
\begin{enumerate}
\item
   \Shaoqing, SD. Zuo, Y.Ren. (2019).
    GISerDaiShaoqing/Urban-Carbon-Dioxide-sources-gridded-maps-and-its-detemination-in-Jinjiang-city 0.3 (Version 0.3) .
    [Data set]. Zenodo. http://doi.org/10.5281/zenodo.3566072.
\end{enumerate}