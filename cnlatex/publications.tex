% AGU style: https://publications.agu.org/agu-grammar-and-style-guide/
\newcommand{\Revision}{\songtix{返修}}
\newcommand{\Review}{\songtix{审稿}}
\newcommand{\Submitted}{\songtix{投稿}}
\newcommand{\CS}{*} % corresponding author
\newcommand{\CF}{\textsuperscript{\#}} % co-first author


\section{\texorpdfstring{\faBook\ 出版物}{出版物}}
\CS 通讯作者, \CF 共同第一作者.

\subsection*{\texorpdfstring{\faBook\ 期刊论文}{期刊论文}}
%\subsection*{Peer-Reviewed Journal Articles}
\begin{enumerate}
%\begin{etaremune}
\item
    窦攀烽, 左舒翟, 任引, {\heiti{戴劭勍}}, 云国梁. (2019).
    气候和土地利用/覆被变化对宁波地区生态系统产水服务的影响.
    {\songtix{环境科学学报}}. \textit{39}(7), 2398-2409.
    \DOI{10.13671/j.hjkxxb.2019.0122}. [CSCD, 复合影响因子=2.544] 
\item
    Q. Yang, TY. Huang, SG. Wang, JS. Li, \Shaoqing, S. Wright, YX. Wang, \& HW. Peng.(2019).
    A GIS-based high spatial resolution assessment of large-scale PV power generation potential and associated CO$_2$ emission reduction in China.
    \textit{Applied Energy}. \textit{247}, 254-269.
    \DOI{10.1016/j.apenergy.2019.04.005}. [SCI, IF = 8.426]
\item
    GL. Yun, YR. He, YT. Jiang, PF. Dou, \Shaoqing. (2019).
    PM2.5 spatiotemporal evolution and drivers in the Yangtze River Delta between 2005 and 2015.
    \textit{Atmosphere}. \textit{10}(2), 55-69.
    \DOI{10.3390/atmos10020055}. [SCI, IF = 2.046]
\item
    SD. Zuo, \Shaoqing, XD. Song, CD. Xu, YL. Liao, WY. Chang, Q. Chen, YY. Li, JF. Tang, W. Man, Y. Ren. (2018).
    Determining the Mechanisms that Influence the Surface Temperature of Urban Forest Canopies by Combining Remote Sensing Methods, Ground Observations, and Spatial Statistical Models. 
    \textit{Remote Sensing}. \textit{10}(11), 1814-1832.
    \DOI{10.3390/rs10111814}. [SCI, IF = 4.118]
\item
    SD. Zuo, \Shaoqing, YY. Li, JF. Tang, Y. Ren. (2018).
	Analysis of Heavy Metal Sources in the Soil of Riverbanks Across an Urbanization Gradient.
    \textit{International Journal of Environmental Research and Public Health}. \textit{15}(10), 2175-2198.
    \DOI{10.3390/ijerph15102175}. [SCI, IF = 2.468]
\item
    {\heiti{戴劭勍}}, 江辉仙, 李佳佳, 苏娴, 吴娟, 任引. (2018).
	H市城区步行环境对两抢一盗警情的影响.
    {\songtix{地理科学}}.  \textit{38}(8), 1235-1244.
    \DOI{10.13249/j.cnki.sgs.2018.08.005}. [CSCD, 复合影响因子=3.964]
\item
    GL. Yun, SD. Zuo, \Shaoqing, XD. Song, CD. Xu, YL. Liao, PQ. Zhao, WY. Chang, Q. Chen, YY. Li, JF. Tang, W. Man, Y. Ren. (2018).
	Individual and Interactive Influences of Anthropogenic and Ecological Factors on Forest PM2.5 Concentrations at an Urban Scale.
    \textit{Remote Sensing}. \textit{10}(4), 521-538.
    \DOI{10.3390/rs10040521}. [SCI, IF = 4.118]
\item
    叶琴, 曾刚, {\heiti{戴劭勍}}, 王丰龙. (2018).
	不同环境规制工具对中国节能减排技术创新的影响——基于285个地级市面板数据.
    {\songtix{中国人口·资源与环境}}. \textit{210}(02), 115-122.
    \DOI{10.12062/cpre.20170915}. [CSCD, 复合影响因子=5.211]
\item
    Q. Ye, \Shaoqing, G. Zeng. (2017).
	Research on the effects of command-and-control and market-oriented policy tools on China’s energy conservation and emissions reduction innovation.
    \textit{Chinese Journal of Population Resources and Environment}. \textit{16}(1), 1-11.
    \DOI{10.1080/10042857.2017.1418273}. [ESCI]
\item
   李苗裔, {\heiti{戴劭勍}}, 王静远, 沈振江. (2016).
	基于政府大数据的北京城市空间发展模式分析.
    {\songtix{地理信息世界}}. \textit{23}(3), 20-26.
    \DOI{10.3969/j.issn.1672-1586.2016.03.004}. [JST, 复合影响因子=0.935]
%\end{etaremune}
\end{enumerate}

%\subsection*{Book Chapter}
\subsection*{\texorpdfstring{\faBook\ 著作章节}{著作章节}}
\begin{enumerate}
%\begin{etaremune}
\item
    GL. Yun, SD. Zuo, \Shaoqing, XD. Song, CD. Xu, YL. Liao, PQ. Zhao, WY. Chang, Q. Chen, YY. Li, JF. Tang, W. Man, Y. Ren. (2019).
	Individual and Interactive Influences of Anthropogenic and Ecological Factors on Forest PM2.5 Concentrations at an Urban Scale.
    \textit{Advances in Quantitative Remote Sensing in China–In Memory of Prof. Xiaowen Li}.316-332.
\item
    \Shaoqing, JJ. Li, SD. Zuo, Y. Ren, HX. Jiang. (2017).
	Landscape-Scale Simulation Analysis of Waterlogging and Sponge City Planning for a Central Urban Area in Fuzhou City, China.
    \textit{International Low Impact Development Conference China 2016 : LID Applications in Sponge City Projects}. 251-260.
    \DOI{10.1061/9780784481042.028}. [EI]
%\end{etaremune}
\end{enumerate}

\subsection*{\texorpdfstring{\faBook\ 会议论文}{会议论文}}
\begin{enumerate}
\item
    \Shaoqing, SD. Zuo, Y. Ren. (2018).
    High-resolution mapping of direct CO$_2$ emissions and uncertainties at the urban scale.
    \textit{Proceedings of Spatial Accuracy 2018}. 88-90. [EI]
\item
    \Shaoqing\CF, XM. Zheng, SD. Zuo,Y. Ren. (2018).
    Improving the prediction accuracy of forest aboveground biomass benchmark map by integrating machine learning and spatial statistics.
    \textit{Proceedings of Spatial Accuracy 2018}. 80-83. [EI]
\item
    SD. Zuo, \Shaoqing, Y. Ren, ZW. Yu. (2017).
    Quantifying the linear and nonlinear relations between the urban form fragmentation and the carbon emission distribution.
    \textit{American Geophysical Union, Fall Meeting 2017, abstract GC21G-1007}.
\end{enumerate}

%\subsection*{Papers submitted/under revision}
\subsection*{\texorpdfstring{\faBook\ 投稿/返修论文}{投稿/返修论文}}
\begin{enumerate}
%\begin{etaremune}
\item
    \Shaoqing, SD. Zuo, Y. Ren.
   A theoretical framework for uncertainty analysis of CO$_2$ emissions gridded maps. 
    (\Submitted).
\item
    \Shaoqing\CF, XM. Zheng, L. Gao, SD. Zuo, Q. Chen, XH Wei, Y. Ren.
    Improving non-representative-sample prediction of forest aboveground biomass maps: A combined machine learning and spatial statistical approach. 
    (\Review).
\item
    SD. Zuo, \Shaoqing, Y. Ren.
    More fragmentized urban form more CO$_2$ emissions? An comprehensive relationship from the combination analysis across different scales 
    (\Revision).
%\end{etaremune}
\end{enumerate}
\subsection*{\texorpdfstring{\faBook\ 专利}{专利}}
%\begin{etaremune}
\begin{enumerate}
\item
     任引, {\heiti{戴劭勍}}, 左舒翟, 郑小曼.
     一种多源数据融合的森林资源清查生物量估算模型.
     ({\songtix{受理}}).
\item
     陈奇, 任引, 郑小曼, 左舒翟, {\heiti{戴劭勍}}.
     一种预测大面积亚热带森林生物量的混合效应模型.
     ({\songtix{实审}}).
%\end{etaremune}
\end{enumerate}

\subsection*{\texorpdfstring{\faBook\ 软件著作权}{软件著作权}}
\begin{enumerate}
\item
   基于环境物联网与 InVEST 的水安全监测系统. V1.0.
    \textit{登记号: 2019SR0517519}. (2019).
\item
    基于环境物联网数据的水源涵养监测系统. V1.0.
    \textit{登记号: 2018SR745897}. (2018).
\end{enumerate}